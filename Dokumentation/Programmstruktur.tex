% !TEX root = Calli.tex
% Kapitelvorlage

\section{Programmstruktur}
\label{sec:Programmstruktur}

Um die einzelnen Früchte sicher detektieren zu können, ist es entscheident zuerst das Ausgangsbild zu filtern und die einzelnen Farben zu extrahieren. Dies geschiet über den HSV - Farbraum. Jede Farbe wird mit Hilfe des Farbwertes (Hue - Wert ), der Farbsättigung (saturation) und der Dunkelstufe / des Hellwertes (value) definiert. Die Beschreibung erfolgt für die unterschiedlichen Farben in folgenden Bereichen:

\begin{center}
\begin{tabular}{ccc}
 & \textbf{Wertebereich} & \textbf{Bereich OpenCV}\\
\textbf{Farbwert} & Farbwinkel auf dem Farbkreis (z.B. 0 $^\circ$ für rot)  & Bereich OpenCV [0,179]\\

\textbf{Farbsättigung} & Sättigung in Prozent (z.B. 0\% Neutralgrau, 100\% reine Farbe) & Bereich OpenCV [0,255]\\

\textbf{Hellwert}  & Helligkeit in Prozent (z.B. 0\% keine Helligkeit, 100\% volle Helligkeit) &  Bereich OpenCV [0,255]\\

\end{tabular}
\end{center}

\textbf{Farbwert}Farbwinkel auf dem Farbkreis (z.B. 0 $^\circ$ für rot)   Bereich OpenCV [0,179]

\textbf{Farbsättigung}Sättigung in Prozent (z.B. 0\% Neutralgrau, 100\% reine Farbe)  Bereich OpenCV [0,255]

\textbf{Hellwert}Helligkeit in Prozent (z.B.100\% volle Helligkeit)   Bereich OpenCV [0,255]


%Ich würde sagen hier ruhig den Farbraum erklären, oder? Sollen wir ein Kapitel Hintergründe machen? Also quasi Hintergrundwissen - ?

 


