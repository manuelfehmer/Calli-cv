% !TEX root = Calli.tex
% Kapitelvorlage

\section{Einleitung}
\label{sec:Einleitung}

Die Anforderungen an die Qualität industriell gefertigter Endprodukte werden immer höher. Um die Kosten der Endkontrolle möglichst gering zu halten, wird eine fehlerresistente und automatisierte Prüfung angestrebt. In vielen Fällen werden dazu Industriekameras eingesetzt. Mit einer angepassten Beleuchtung, der entsprechenden Ausrichtung und einer ausgereiften Software ersetzen sie mittlerweile viele manuelle Kontrollen. Insbesondere zur Farb- und Formerkennung sind sie gut geeignet.

Das bearbeitete Thema hat zum Gegenstand eine Farb- und Formerkennung mit einer Kamera zu realisieren. Grundlage ist das Kinder - Reaktionsspiel Halli Galli. 

\subsection{Halli Galli - Spielanleitung}

Halli Galli ist ein Reaktionskartenspiel für Kinder. Obwohl das Spielprinzip sehr einfach ist, ist es aufgrund der hohen Spielgeschwindigkeit, der geforderten Konzentrationsfähigkeit sowie der Stressresistenz bei vielen Erwachsenen beliebt. 

\textbf{Ziel des Spiels} 




to do 

1.) Theoretische Grundlagen (z.B. Messprinzip)

2.) Genaue Beschreibung des Geräteaufbaus (Bilder)

3.) Beschreibung der Halcon-Programmstruktur und die Funktionsweise der Programmblöcke

4.) Gut dokumentierter Halcon-Code 

5.) Ergebnisse 

6.) Testbilder, um das Programm auch ohne Kamera testen zu können.

Die Doku auf CD mit allen Bilder muss dem gedruckten Bericht beigelegt werden.
