% !TEX root = Calli.tex
% Kapitelvorlage

\section{Anleitung}
\label{sec:Anleitung}

\subsection{Anwendung mit Kamera-Stream}
Um das Programm mit dem geringsten Aufwand ausführen zu können, wurde mit dem \emph{PyInstaller} aus dem Python-Programm eine .exe - Anwendung erstellt. Diese Version ist auf eine Webcam eingestellt, welche 40~cm über dem Spielaufbau positioniert ist, siehe Kapitel \ref{sub:Aufbau}.


\subsection{Konsolenanwendung für Bilder}
Für Demonstrationszwecke wurde eine Variante als Konsolenanwendung erstellt, um das Programm ohne Kamera mit Beispielbildern zu testen. 
An die Anwendung wird mit dem Kürzel \lstinline{-i} die Bilddatei angehängt.
 
Wahlweise können mit Optionskürzeln zusätzliche Informationen angezeigt werden:
\begin{itemize}
\item \lstinline{-c} zeigt in Weiß die Kontur der umschließenden Rechtecke an. 
\item \lstinline{-o} zeigt die Textausgabe der Werte in der Formsegmentierung an.
\begin{itemize}
\item Verhältnis der detektierten Fläche zur Fläche des umschließenden Rechteckes
\item Verhältnis der Höhe zur Breite des umschließenden Rechteckes
\item Größe der detektierten Fläche
\end{itemize}
\end{itemize}
\begin{lstlisting}[ ]
cd <programdir>
Calli-Cv-Pictures.exe -i Testbild01.PNG -c -o
\end{lstlisting}
